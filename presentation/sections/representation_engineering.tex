\section{Representation Engineering for Shortcut Learning}

\begin{frame}{Representation Engineering (RepE)}
\begin{columns}
\begin{column}{0.4\textwidth}
What if we could \textbf{understand} and \textbf{manipulate} LLMs through their \highlight{\textbf{hidden representations}}?\\
\vspace{2em}
\emph{Zou et al, "Representation Engineering: A Top-Down Approach to AI Transparency", 2025}
\end{column}
\begin{column}{0.5\textwidth}
\includegraphics[width=\textwidth]
{assets/zou_fairness}
\end{column}
\end{columns}
\end{frame}

\begin{frame}{Representation Engineering (RepE)}
\begin{columns}
\begin{column}{0.4\textwidth}
What if we could \textbf{understand} and \textbf{manipulate} LLMs through their \highlight{\textbf{hidden representations}}?\\
\vspace{2em}
\emph{Zou et al, "Representation Engineering: A Top-Down Approach to AI Transparency", 2025}
\end{column}
\begin{column}{0.5\textwidth}
\includegraphics[width=0.9\textwidth]
{assets/zou_power}
\end{column}
\end{columns}
\end{frame}


\begin{frame}{Representation Engineering (RepE)}
\begin{center}
\includegraphics[width=0.9\textwidth]{assets/RepE_overview_no_title.png}
\end{center}
\end{frame}

\begin{frame}{Representation Engineering (RepE)}
\begin{center}
    Azaria and Mitchell, "The Internal State of an LLM Knows When It's Lying", 2023
\end{center}
\vspace{2em}
\begin{quote}
\centering
Then, \textbf{if} an LLM \highlight{knows} when it's taking a shortcut, \\
we can use Representation Engineering to detect it and suppress it.
\end{quote}
\end{frame}

\begin{frame}{Idea: RepE for Shortcut Mitigation}
\begin{itemize}
    \item Collect \textbf{contrastive pairs} $(P^{(i)}_{+shortcut}, P^{(i)}_{-shortcut})$ of textual prompts\\
    that differ only in the presence of a shortcut cue
    \vspace{1em}
    \pause
    \item Use \textbf{Representation Reading} to extract a latent linear direction\\
    corresponding to \emph{shortcut reliance}
    \vspace{1em}
    \pause
    \item Use \textbf{Representation Control} to suppress shortcut-driven behavior\\
    in the model at inference time
\end{itemize}
\end{frame}

\begin{frame}{RepE-based framework for Shortcut Mitigation}
\begin{columns}
\begin{column}{0.2\textwidth}
\begin{itemize}
    \item Data Pre-processing
    \item RepReading
    \item RepControl
\end{itemize}
\end{column}
\begin{column}{0.8\textwidth}
\begin{center}
    \includegraphics[height=\textheight]{assets/framework}
\end{center}
\end{column}
\end{columns}
\end{frame}

\begin{frame} {Data Pre-Processing}
    \includegraphics[width=0.9\textwidth]{assets/preprocessing.png}
\end{frame}

\begin{frame} {Representation Reading}
    \begin{columns}
        \begin{column}{0.75\textwidth}
            \includegraphics[width=\textwidth]{assets/RepReading.png}
        \end{column}
        \begin{column}{0.25\textwidth}
            $V_l$ represents a \emph{shortcut reliance} direction for \\the layer $l$\\
            \vspace{2em}
            Can be used for \highlight{shortcut detection}
        \end{column}
    \end{columns}
    
\end{frame}

\begin{frame} {Representation Control}
    \begin{columns}
        \begin{column}{0.5\textwidth}
            \begin{center}
                \includegraphics[width=0.9\textwidth]{assets/RepControl.png}
            \end{center}
            
        \end{column}
        \begin{column}{0.5\textwidth}
                \begin{itemize}
        \item \textit{Linear Combination}: 
        \begin{equation*} 
            H_l' = H_l + \alpha \textcolor{sintefdarkgreen}{S_l V_l} 
        \end{equation*}
        \item \textit{Piece-wise Operation}: 
        \begin{equation*} 
            H_l' = H_l + \alpha \textcolor{sintefdarkgreen}{S_l}\ sign(H_l^T\textcolor{sintefdarkgreen}{V_l})\textcolor{sintefdarkgreen}{V_l} \end{equation*}
        \item \textit{Projection}: 
        \begin{equation*} 
            H_l' = \left( I - \frac{\textcolor{sintefdarkgreen}{V_lV_l}^T}{\|\textcolor{sintefdarkgreen}{V_l}\|^2}\right) H_l 
        \end{equation*}
    \end{itemize}
        \end{column}
    \end{columns}
\end{frame}


\begin{frame}{Design Pros and Cons}
    \begin{columns}
        \begin{column}{0.4\textwidth}
            \begin{center}
                  \begin{itemize}
            \item[\textcolor{sintefdarkgreen}{\cmark}] \textbf{Training-free} approach
            \item[\textcolor{sintefdarkgreen}{\cmark}] \textbf{Modular}, trasferable across models and tasks
            \item[\textcolor{sintefdarkgreen}{\cmark}] \textbf{Enhances transparency} of model's behavior
        \end{itemize}
            \end{center}
        \end{column}
        \begin{column}{0.4\textwidth}
            \begin{center}
                  \begin{itemize}
            \item[\xmark] Requires \textbf{carefully crafted data} 
            \item[\xmark] Can only target \textbf{open-weights} models
            \item[\xmark] \textbf{Assumes existence and linearity} of a shortcut reliance direction in the latent space
        \end{itemize}
            \end{center}
        \end{column}
    \end{columns}
 
\end{frame}
