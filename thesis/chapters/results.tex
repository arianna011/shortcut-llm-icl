\documentclass[../main.tex]{subfiles}
% !TeX root = ../main.tex
\graphicspath{{\subfix{../assets/}}}
\begin{document}

\chapter{Experimental Evaluation}

Several experiments were conducted to evaluate the effectiveness of the proposed RepE-based framework for shortcut detection and mitigation.  
Section~\ref{sec:setup} describes the experimental setup, including the involved models, datasets and implementation details, while Section~\ref{sec:results} presents and discusses the obtained results. The experiments aim to determine whether latent shortcut directions can be reliably extracted from LLM representations and whether their manipulation through RepControl leads to measurable improvements in model behavior under In-Context Learning (ICL).

\section{Experimental Setup}
\label{sec:setup}

\subsection{Models}
\subsection{Datasets}
\subsection{Prompt Design and ICL Configuration}
% setup ----
% model
% training dataset 
% test dataset (size)
% ICL configuration, method configurations
% shortcut suite method
% iid vs ood testing

\section{Results and Analysis}
\label{sec:results}

\subsection{Shortcut Detection Results}
\subsection{Shortcut Mitigation Results}
\subsection{Qualitative Analysis}

\subbib{}
\end{document}